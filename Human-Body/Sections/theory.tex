El modelo biomecánico utilizado describe el cuerpo humano como un sistema
de cuatro masas conectadas entre sí mediante resortes y amortiguadores. Este
enfoque permite representar de forma simplificada la respuesta dinámica del
cuerpo frente a las vibraciones transmitidas desde un asiento vehicular. Las
ecuaciones diferenciales que gobiernan el comportamiento del sistema se
presentan en las \cref{eq:m1,eq:m2,eq:m3,eq:m4}, donde cada masa modela
una parte específica del cuerpo humano, tal como se describe a continuación:

\begin{itemize}
  \item \( m_1 \): Torso inferior.
  \item \( m_2 \): Víscera.
  \item \( m_3 \): Torso superior.
  \item \( m_4 \): Cabeza y el cuello.
\end{itemize}

\begin{align}
  m_1 \ddot{x}_1 &= -c_2 \left( \dot{x}_1 - \dot{x}_2 \right) - k_2 \left( x_1
  - x_2 \right) \label{eq:m1} \\
    &\quad - c_{31} \left( \dot{x}_1 - \dot{x}_2 \right) + k_{31} \left( x_1
    - x_3 \right) \nonumber \\
    &\quad + c_1 \left( \dot{x}_s - \dot{x}_1 \right) + k_1 \left( x_s -
    x_1 \right) \nonumber \\
  m_2 \ddot{x}_2 &= c_2 \left( \dot{x}_1 - \dot{x}_2 \right) + k_2 \left( x_1 -
  x_2 \right) \label{eq:m2} \\
    &\quad - c_3 \left( \dot{x}_2 - \dot{x}_3 \right) - k_3 \left( x_2 -
    x_3 \right) \nonumber \\
  m_3 \ddot{x}_3 &= c_3 \left( \dot{x}_2 - \dot{x}_3 \right) + k_3 \left( x_2 -
  x_3 \right) \label{eq:m3} \\
    &\quad + c_{31} \left( \dot{x}_1 - \dot{x}_3 \right) + k_{31} \left( x_1
    - x_3 \right) \nonumber \\
    &\quad - c_4 \left( \dot{x}_3 - \dot{x}_4 \right) - k_4 \left( x_3 -
    x_4 \right) \nonumber \\
  m_4 \ddot{x}_4 &= c_4 \left( \dot{x}_3 - \dot{x}_4 \right) + k_4 \left( x_3
  - x_4 \right) \label{eq:m4}
\end{align}

El desplazamiento del asiento \( x_s \) genera las fuerzas que provocan el
movimiento en el sistema, afectando las masas a través de resortes y
amortiguadores, los cuales introducen efectos de rigidez y amortiguamiento,
respectivamente. Cada ecuación refleja las interacciones entre las masas,
modeladas por los coeficientes de rigidez (\( k \)) y los coeficientes de
amortiguamiento (\( c \)), que permiten simular la propagación de vibraciones
a través del cuerpo humano. Estas ecuaciones son fundamentales para estudiar
cómo el cuerpo responde a las fuerzas transmitidas por el vehículo en condiciones
de vibración, y son cruciales para el diseño de sistemas que mejoren la seguridad
y confort del usuario en un entorno vehicular.
