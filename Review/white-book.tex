\documentclass[journal, table]{IEEEtran}
\usepackage[english, spanish]{babel}
\usepackage[sorting=none]{biblatex}
\bibliography{ref.bib}
\usepackage{amsmath, amsfonts, amsthm}
\usepackage{hyperref, url}
\usepackage{graphicx}
\usepackage{fancyhdr, last page}
\usepackage{siunitx}
\usepackage{anyfontsize}
\usepackage{csquotes}
\usepackage{xcolor}
\usepackage[affil-sl]{authblk}
\usepackage[spanish]{cleveref}
\AtBeginDocument{\decimalpoint}

\hypersetup{
    colorlinks=true,
    linkcolor=black,
    urlcolor=blue,
    pdftitle={Review Control White Book - JLAM}
}
\urlstyle{same}
\sisetup{separate-uncertainty}

\begin{document}

\title{\textbf{El Control Automático} \\
    \small{Reseña de los Capítulos 2 y 3 del Libro blanco del control automático
\cite{de-automatica-2009}} \\ \today}

\author[*]{Julian L. Avila
    \thanks{Julian Avila: 20212107030}}

\affil[*]{Programa Académico de Física\\
    Universidad Distrital Francisco José de Caldas}

\markboth{}
{Shell \MakeLowercase{\textit{et al.}}: Bare Demo of IEEEtran.cls for IEEE Journals}

\maketitle

\selectlanguage{english}
\begin{abstract}
\end{abstract}

\begin{IEEEkeywords}
Digital design, logic gates, boolean algebra
\end{IEEEkeywords}

\selectlanguage{spanish}
\begin{abstract}
\end{abstract}
\begin{IEEEkeywords}
diseño digital, compuertas lógicas, álgebra booleana
\end{IEEEkeywords}

\tableofcontents

\section{El Control Automático}%
\label{sec:El Control Automático}

El control automático es un concepto fundamental en la ingeniería, presente en
casi todas las situaciones cotidianas. Se enfoca en diseñar sistemas que operen
de manera autónoma y prácticamente instantánea. Un sistema de control automático
toma decisiones basadas en algoritmos, que definen las acciones necesarias para
alcanzar objetivos específicos.
La electrónica actúa como el soporte físico de estos algoritmos, permitiendo su
implementación y ejecución.

Un ejemplo común es el control de velocidad de un coche. Aquí, el sistema regula
automáticamente la velocidad del vehículo, ajustando la aceleración o el frenado
según sea necesario para mantener una velocidad predefinida, sin intervención
directa del conductor.

\subsection{La realimentación}%
\label{sec:La realimentación}

La realimentación es un proceso clave en el control automático, ya que consiste
en utilizar la medida de la variable a controlar para tomar decisiones más
precisas.
Esto permite corregir errores del modelo y mejorar la precisión en la
consecución de los objetivos.
Sin la realimentación, en muchos casos no sería posible obtener resultados
cercanos al objetivo.

Sin embargo, la realimentación también implica desafíos, como la necesidad de
estimar o medir con precisión las variables involucradas.
Un problema crítico es que la información estimada puede no ser fiable o estar
corrupta, lo que afecta la eficacia del control.
Para mitigar este riesgo, es esencial contar con redundancia en las mediciones,
asegurando así la fiabilidad de los datos utilizados para la toma de decisiones.

\printbibliography
\end{document}

