\selectlanguage{english}

\begin{abstract}

This study examines the dynamic response of a human body subjected to
vibrations in a vehicle seat, employing the biomechanical model proposed by  
Wan Schimmels. The system's response to a unit step input is analysed to  
determine displacement, velocity, and acceleration across various body  
segments. The findings reveal significant variations in dynamic behaviour,  
with the highest accelerations observed in the head and neck regions due to  
increased stiffness and damping coefficients. These results emphasise the need  
to optimise mechanical parameters to improve user comfort and safety in  
vehicular environments.  

\end{abstract}

\begin{IEEEkeywords}
  Dynamic response; Vehicle seat; Biomechanical model.
\end{IEEEkeywords}

\selectlanguage{spanish}
\begin{abstract}

Este estudio examina la respuesta dinámica de un cuerpo humano sometido a
vibraciones en un asiento vehicular, utilizando el modelo biomecánico propuesto
por Wan Schimmels. Se analiza la respuesta del sistema a una entrada de escalón
unitario para determinar el desplazamiento, la velocidad y la aceleración en
diferentes segmentos del cuerpo. Los resultados muestran variaciones significativas
en el comportamiento dinámico, con las mayores aceleraciones observadas en la
cabeza y el cuello debido a los mayores coeficientes de rigidez y amortiguamiento
en estas regiones. Estos hallazgos subrayan la importancia de optimizar los
parámetros mecánicos para mejorar el confort y la seguridad del usuario en
entornos vehiculares.

\end{abstract}

\begin{IEEEkeywords}
  Respuesta dinámica, Asiento vehicular, Modelo biomecánico.
\end{IEEEkeywords}
