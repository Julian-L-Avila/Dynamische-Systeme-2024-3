Empleando las ecuaciones diferenciales que describen el sistema, se obtuvo la
respuesta a una entrada \( x_s \) de escalón unitario con amplitud
\qty{0.01}{\metre}, tiempo de activación \( t = \qty{1}{\second} \) y duración
de \qty{5}{\second}.

Las \cref{fig:x1,fig:x2,fig:x3,fig:x4} presentan el desplazamiento de cada una
de las masas del modelo. El mayor pico ocurre en la víscera, mientras que el
torso inferior exhibe un desplazamiento semejante a un sistema críticamente
amortiguado.

Las \cref{fig:v1,fig:v2,fig:v3,fig:v4} ilustran las velocidades de las masas.
Como se deduce de la \cref{fig:x4}, la cabeza y el cuello alcanzan las mayores
velocidades, resultado de la combinación de un alto \( k_4 \) y un valor medio
de \( c_4 \).

Las \cref{fig:a1,fig:a2,fig:a3,fig:a4} muestran las aceleraciones de las masas.
Destacan picos elevados, siendo el más alto en el cuello y la cabeza, con más
de \qty{24.0}{\metre\per\square\second}. Este valor podría ser incómodo para el
usuario. Las aceleraciones aumentan desde la base hacia la parte superior del
cuerpo, lo cual es coherente con la propagación del movimiento.


\begin{figure}[htbp!]
  \centering
  \resizebox{\linewidth}{!}{
    \begin{tikzpicture}
	\begin{axis}[
		title={Desplazamiento \( x_1 \)},
		width=12cm,
		height=8cm,
		xlabel={\(t \, [\si{\second}]\)},
		ylabel={\(x_1 \, [\si{\milli\metre}]\)},
		grid=both,
		grid style=dashed,
		legend style={font=\small},
		every axis plot/.append style={thick},
		xmin=0,
		xmax=5,
		x tick label style={/pgf/number format/.cd, fixed, fixed zerofill, precision=1},
		y tick label style={/pgf/number format/.cd, fixed, fixed zerofill, precision=1},
		ytick scale label code/.code={},
		ytick distance=2/1000,
		xtick distance=1,
		minor y tick num=1,
		minor x tick num=2,
	]
	\addplot[
		color=Plum,
		line width=1.2pt,
	]
	table[col sep=tab, x index=0, y index=1]{./Data/system_solution_with_velocities_and_accelerations.csv};
	\end{axis}
\end{tikzpicture}

  }
  \caption{Desplazamiento del Torso inferior.}
  \label{fig:x1}
\end{figure}

\begin{figure}[htbp!]
  \centering
  \resizebox{\linewidth}{!}{
    \input{./Figures/x-2-plot.tex}
  }
  \caption{Desplazamiento de la Víscera.}
  \label{fig:x2}
\end{figure}

\begin{figure}[htbp!]
  \centering
  \resizebox{\linewidth}{!}{
    \input{./Figures/x-3-plot.tex}
  }
  \caption{Desplazamiento del Torso superior.}
  \label{fig:x3}
\end{figure}

\begin{figure}[htbp!]
  \centering
  \resizebox{\linewidth}{!}{
    \input{./Figures/x-4-plot.tex}
  }
  \caption{Desplazamiento de la Cabeza y Cuello.}
  \label{fig:x4}
\end{figure}

\begin{figure}[htbp!]
  \centering
  \resizebox{\linewidth}{!}{
    \input{./Figures/v-1-plot.tex}
  }
  \caption{Velocidad del Torso inferior.}
  \label{fig:v1}
\end{figure}

\begin{figure}[htbp!]
  \centering
  \resizebox{\linewidth}{!}{
    \input{./Figures/v-2-plot.tex}
  }
  \caption{Velocidad de la Víscera.}
  \label{fig:v2}
\end{figure}

\begin{figure}[htbp!]
  \centering
  \resizebox{\linewidth}{!}{
    \begin{tikzpicture}
	\begin{axis}[
		title={Velocidad \( \dot{x}_3 \)},
		width=12cm,
		height=8cm,
		xlabel={\(t \, [\si{\second}]\)},
		ylabel={\(\dot{x}_3 \, [\si{\centi\metre\per\second}]\)},
		grid=both,
		grid style=dashed,
		legend style={font=\small},
		every axis plot/.append style={thick},
		xmin=0,
		xmax=5,
		x tick label style={/pgf/number format/.cd, fixed, fixed zerofill, precision=1},
		y tick label style={/pgf/number format/.cd, fixed, fixed zerofill, precision=1},
		ytick scale label code/.code={},
		ytick distance=6,
		xtick distance=1,
		minor y tick num=1,
		minor x tick num=2,
	]
	\addplot[
		color=Green,
		line width=1.2pt,
	]
	table[col sep=tab, x expr=\thisrowno{0}, y expr=100*\thisrowno{8}]{./Data/system_solution_with_velocities_and_accelerations.csv};
	\end{axis}
\end{tikzpicture}


  }
  \caption{Velocidad del Torso superior.}
  \label{fig:v3}
\end{figure}

\begin{figure}[htbp!]
  \centering
  \resizebox{\linewidth}{!}{
    \input{./Figures/v-4-plot.tex}
  }
  \caption{Velocidad de la Cabeza y Cuello.}
  \label{fig:v4}
\end{figure}

\begin{figure}[htbp!]
  \centering
  \resizebox{\linewidth}{!}{
    \input{./Figures/a-1-plot.tex}
  }
  \caption{Aceleración del Torso inferior.}
  \label{fig:a1}
\end{figure}

\begin{figure}[htbp!]
  \centering
  \resizebox{\linewidth}{!}{
    \input{./Figures/a-2-plot.tex}
  }
  \caption{Aceleración de la Víscera.}
  \label{fig:a2}
\end{figure}

\begin{figure}[htbp!]
  \centering
  \resizebox{\linewidth}{!}{
    \input{./Figures/a-3-plot.tex}
  }
  \caption{Aceleración del Torso superior.}
  \label{fig:a3}
\end{figure}

\begin{figure}[htbp!]
  \centering
  \resizebox{\linewidth}{!}{
    \begin{tikzpicture}
	\begin{axis}[
		title={Acelaración \( \ddot{x}_4 \)},
		width=12cm,
		height=8cm,
		xlabel={\(t \, [\si{\second}]\)},
		ylabel={\(\ddot{x}_4 \, [\si{\metre\per\square\second}]\)},
		grid=both,
		grid style=dashed,
		legend style={font=\small},
		every axis plot/.append style={thick},
		xmin=0,
		xmax=5,
		x tick label style={/pgf/number format/.cd, fixed, fixed zerofill, precision=1},
		y tick label style={/pgf/number format/.cd, fixed, fixed zerofill, precision=1},
		ytick scale label code/.code={},
		ytick distance=3,
		xtick distance=1,
		minor y tick num=1,
		minor x tick num=2,
	]
	\addplot[
		color=Cyan,
		line width=1.2pt,
	]
	table[col sep=tab, x expr=\thisrowno{0}, y expr=\thisrowno{12}]{./Data/system_solution_with_velocities_and_accelerations.csv};
	\end{axis}
\end{tikzpicture}


  }
  \caption{Aceleración de la Cabeza y Cuello.}
  \label{fig:a4}
\end{figure}

